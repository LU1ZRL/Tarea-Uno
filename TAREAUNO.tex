\documentclass[11pt,letterpaper]{article}

%----- Configuración del estilo del documento------%
\usepackage{graphicx}
\usepackage[table]{xcolor}
\usepackage[left=2cm,right=2cm,top=1.8cm,bottom=2.3cm]{geometry}
\usepackage{fancyhdr}
\usepackage{lastpage}
\usepackage[spanish]{babel}
\pagestyle{fancy}
\fancyhf{}
\rfoot{\textit{Página \thepage \hspace{1pt} de \pageref{LastPage}}}

%------ Paquetes matemáticos básicos --------%
\usepackage{amsmath, amssymb, amsthm}

%------ Texto aleatorio y listas ----- %
\usepackage{lipsum, enumitem}

\begin{document}

%------ Encabezado -------- %
\begin{center}
    \begin{minipage}{3cm}
    	\begin{center}
    		\includegraphics[height=3.4cm]{imagenes/logo_unam.png}
    	\end{center}
    \end{minipage}\hfill
    \begin{minipage}{10cm}
    	\begin{center}
    	\textbf{\large Universidad Nacional Autónoma de México}\\[0.1cm]
        \textbf{Facultad de Ciencias}\\[0.1cm]
        \textbf{Matemáticas para las Ciencias Aplicadas 2 | Grupo 7056}\\[0.1cm]
        \textbf{Tarea 1 }\\[0.1cm]
        Cisneros Álvarez Danjiro\\[0.1cm]
        Rodríguez López Luis Fernando\\[0.1cm]
        Tenorio Reyes Ihebel Luro\\[0.1cm]
        21/02/2025
    	\end{center}
    \end{minipage}\hfill
    \begin{minipage}{3cm}
    	\begin{center}
    		\includegraphics[height=3.4cm]{imagenes/Logo_FC.png}
    	\end{center}
    \end{minipage}
\end{center}

\noindent\rule{\linewidth}{0.4pt}

%------ Fin de encabezado -------- %

%\section*{1ra Parte}

\subparagraph{Ejercicios: Sección 11.2 Anton-Bivens-Davis (pp. 782-784).}

% ---- 01. Ejercicio 52 DANJIRO ---- %
\section{Ejercicio 52, Sección 11.2}

% ---- 02. Ejercicio 56 LUIS ---- %
\section{Ejercicio 56, Sección 11.2}
Un bloque con un peso de 100 N está suspendido por cables A y B, como se muestra en la figura adjunta.
\begin{enumerate}
    \item Utilice una herramienta gráfica para graficar las fuerzas que el bloque ejerce a lo largo de los cables A y B como funciones del ``hundimiento'' $d$.
    \item ¿El aumento del hundimiento incrementa o disminuye las fuerzas en los cables?
    \item ¿Cuánto hundimiento se requiere si los cables no pueden tolerar fuerzas superiores a 150 N?
\end{enumerate}

\begin{figure}[h]
    \centering
    \includegraphics[width=0.2\textwidth]{imagenes/Figure_Ex-55.png}
    \hspace{5cm}
    \includegraphics[width=0.2\textwidth]{imagenes/Figure_Ex-56.png}
\end{figure}


% ---- 03. Ejercicio 58 IHEBEL ---- %
\section{Ejercicio 58, Sección 11.2}


\subparagraph{Ejercicios: Sección 11.3 Anton-Bivens-Davis (pp. 792-794).}

% ---- 04. Ejercicio 19 LUIS ---- %
\section{Ejercicio 19, Sección 11.3}
La figura adjunta muestra un cubo.
\begin{enumerate}
    \item Encuentre el ángulo entre los vectores $\mathbf{d}$ y $\mathbf{u}$ al grado más cercano.
    \item Haga una conjetura sobre el ángulo entre los vectores $\mathbf{d}$ y $\mathbf{v}$, y confirme su conjetura calculando el ángulo.
\end{enumerate}

\begin{figure}[h]
    \centering
    \includegraphics[width=0.2\textwidth]{imagenes/Figure_Ex-18.png}
    \hspace{5cm}
    \includegraphics[width=0.2\textwidth]{imagenes/Figure_Ex-19.png}
\end{figure}


% ---- 05. Ejercicio 34 IHEBEL ---- %
\section{Ejercicio 34, Sección 11.3}
Como se muestra en la figura adjunta, un niño con masa de 34 kg está sentado en un tobogán de juegos suave (sin fricción) que está inclinado en un ángulo de $27^\circ$ con la horizontal. Estime la fuerza que el niño ejerce sobre el tobogán, y estime cuánta fuerza debe aplicarse en la dirección de $\mathbf{P}$ para evitar que el niño se deslice hacia abajo por el tobogán. Tome la aceleración debida a la gravedad como 9.8 m/s$^2$.


% ---- 06. Ejercicio 35 DANJIRO ---- %
\section{Ejercicio 35, Sección 11.3}
Para el niño del Ejercicio 34, estime cuánta fuerza debe aplicarse en la dirección de $\mathbf{Q}$ (mostrada en la figura adjunta) para evitar que el niño se deslice hacia abajo por el tobogán.

\begin{figure}[h]
    \centering
    \includegraphics[width=0.2\textwidth]{imagenes/Figure_Ex-34.png}
    \hspace{5cm}
    \includegraphics[width=0.2\textwidth]{imagenes/Figure_Ex-35.png}
\end{figure}

% ---- 07. Ejercicio 36 LUIS ---- %
\section{Ejercicio 36, Sección 11.3}

Suponga que el tobogán en el Ejercicio 34 tiene 4 m de largo. Estime el trabajo realizado por la gravedad si el niño se desliza desde la parte superior del tobogán hasta la parte inferior.


\subparagraph{Ejercicios: Sección 11.4 Anton-Bivens-Davis (pp. 803-805).}

% ---- 08. Ejercicio 40 IHEBEL ---- %
\section{Ejercicio 40, Sección 11.4}

% ---- 09. Ejercicio 41 DANJIRO ---- %
\section{Ejercicio 41, Sección 11.4}

% ---- 10. Ejercicio 49 POR SORTEAR ---- %
\section{Ejercicio 49, Sección 11.4}
Use un CAS para aproximar el área mínima de un triángulo si dos de sus vértices son $(2, -1, 0)$ y $(3, 2, 2)$ y su tercer vértice está en la curva $y = \ln x$ en el plano $xy$.

\end{document}